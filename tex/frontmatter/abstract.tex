%% Direttive TeXworks:
% !TeX root = ../../semprini_luca_tesi.tex
% !TEX encoding = UTF-8 Unicode
% !TEX program = arara
% !TEX TS-program = arara
% !TeX spellcheck = it-IT

\begin{abstract}
Lo scopo della presente tesi di laurea è quello di presentare in maniera esaustiva ed approfondita le caratteristiche e le peculiarità fondamentali del linguaggio di programmazione Kotlin, con particolare attenzione alle sue applicazioni per lo sviluppo di sistemi Android.\\

La trattazione è strutturata su quattro capitoli: nel \Cref{ch:presentazione} viene illustrato il contesto storico in cui la progettazione del linguaggio ha avuto luogo, nonché le piattaforme target per cui poterlo utilizzare, i concetti fondamentali della filosofia del linguaggio e i principali metodi di compilazione; nel \Cref{ch:sintassi} viene esposta la sintassi basilare del linguaggio e le sue caratteristiche più interessanti, fornendo in maniera sistematica esempi di codice esplicativi;
nel \Cref{ch:android} viene approfondita la piattaforma Android come target principale di Kotlin, fornendo esempi di feature proprie del linguaggio che possono risultare congeniali allo sviluppo di applicazioni in questo ambiente: anch'esse vengono corredate da esempi di codice per un approfondimento ulteriore; in questo capitolo vengono, inoltre, analizzate le prestazioni del linguaggio sotto l'aspetto della velocità di compilazione, del conteggio dei metodi, dell'impatto sulle dimensioni dell'applicazione scritta in Kotlin, e sulla stabilità effettiva di quest'ultima. Infine, nel \Cref{ch:conclusione} viene effettuata un'analisi sullo stato attuale del linguaggio e la sua popolarità, concludendo con una valutazione delle prospettive future in ambito Android che Kotlin potrà riscontrare nei mesi a venire.


\end{abstract}
