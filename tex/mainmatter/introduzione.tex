%% Direttive TeXworks:
% !TeX root = ../semprini_luca_tesi.tex
% !TEX encoding = UTF-8 Unicode
% !TEX program = arara
% !TEX TS-program = arara
% !TeX spellcheck = it-IT

\chapter*{Introduzione}\label{ch:intro}
\addcontentsline{toc}{chapter}{\nameref{ch:intro}}

Il lavoro svolto in questa tesi di laurea ha come obiettivo quello di illustrare ed analizzare le caratteristiche principali del linguaggio di programmazione Kotlin e le sue funzionalità più vantaggiose per lo sviluppo di applicazioni Android.\\

Kotlin è un linguaggio di programmazione orientato agli oggetti staticamente tipizzato che nasce nel 2011, anno del primo annuncio ufficiale, da un team di sviluppatori di JetBrains. Si tratta di un linguaggio prevalentemente improntato su Java, che viene quindi compilato in un byte-code eseguibile sulla Java Virtual Machine; tuttavia, offre la possibilità di essere compilato in JavaScript o in codice nativo, utilizzando la struttura di compilazione LLVM \cite{LLVMWiki}.\\
Kotlin si pone oggi come obiettivo quello di rappresentare una valida alternativa a Java soprattutto nello sviluppo di applicazioni Android, fornendo un sistema di tipi sicuro dal punto di vista della prevenzione di crash durante l'esecuzione e una sintassi moderna, concisa ed elegante, che riprende molti noti costrutti che hanno fatto la fortuna di linguaggi come Scala, Groovy, C\# e Java stesso.\\
La libreria standard di Kotlin si appoggia completamente sul framework Java, estendendolo con funzionalità inedite: questo dimostra la forte interoperabilità tra i due linguaggi.
Kotlin presenta diverse caratteristiche sintattiche di rilievo, tra cui la possibilità di definire funzioni {\em inline}, se costituite da un'unica espressione, che rendono il byte-code generato dal compilatore molto più performante; le data classes sono un costrutto che permette di definire in pochissime linee di codice una classe che fornisca un supporto conveniente per immagazzinare e gestire dati; le funzioni di estensione sono funzioni che possono essere chiamate come membro di una certa classe ma vengono definite al di fuori di essa, permettendo di estendere codice scritto in qualsiasi linguaggio JVM.\\

L'utilizzo di Kotlin per la programmazione di applicazioni Android sta prendendo sempre più piede, grazie a diverse funzionalità che risultano peculiari per questo ambiente di esecuzione, come le Lambda Expressions (simili alle omonime Java, ma più potenti), le Coroutines (che utilizzano i paradigmi più diffusi di programmazione asincrona d C\#, Python, Go e JavaScript) o le Annotazioni. Un'altra importante discriminante che ha reso Kotlin sempre più popolare per lo sviluppo su Android è stato l'annuncio ufficiale da parte di Google di renderlo un linguaggio ufficialmente supportato (e integrato nella IDE ufficiale di programmazione Android Studio).\\

Il lavoro svolto ha riguardato in primis l'apprendimento del linguaggio e lo studio della sua evoluzione nel tempo, oltre che all'analisi approfondita di tutte le sue sfaccettature e la raccolta delle opinioni degli esperti. Si è continuato organizzando la stesura in maniera originale, cercando di seguire una linea coerente con l'intenzione di toccare in maniera graduale tutti gli argomenti più significativi che riguardassero il liguaggio in questione. Gli esempi di codice forniti nel corso dell'elaborato sono atti a contribuire in maniera sistematica alla presentazione del linguaggio e sono stati concepiti in maniera da risultare sufficientemente esplicativi, sulla base dell'esperienza personale con il linguaggio analizzato.\\
