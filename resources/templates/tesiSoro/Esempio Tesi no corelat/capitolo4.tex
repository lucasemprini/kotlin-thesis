
\chapter{Conclusioni}

In questo elaborato \'e stata analizzata l'evoluzione che il web ha subito nel tempo, in particolare la trasformazione dei contenuti statici in dinamici grazie a teconologie come HTML5 e JavaScript che hanno permesso un livello notevolmente maggiore di interazione tra la pagina e l'utente.

Grazie a questa importante innovazione \'e stato possibile sviluppare vere e proprie applicazioni sfruttando il web come piattaforma. Inizialmente erano suddivise in pi\'u pagine ma, grazie ai framework come AngularJS, si riusc\'i ad implementare le intere applicazioni su una sola pagina.
Questi particolari framework si basano tutt'ora su JavaScript, ma quest'ultimo presenta ormai problemi impossibili da risolvere con una sua ulteriore evoluzione.
Per questo motivo sono stati introdotti dei nuovi linguaggi di alto livello con lo scopo di sostituire JavaScript nello sviluppo di applicazione web.

Dart \'e stato il primo linguaggio di questo tipo ed essere prodotto; offre notevoli vantaggi sia in termini di performance che di funzionalit\'a, ma presenta anche alcune limitazioni, di cui la pi\'u importante riguarda la necessit\'a di essere eseguito su una particolare virtual machine, che al giorno d'oggi \'e poco diffusa tra i browser, per mentenere alte le sue performance.

Un secondo nuovo linguaggio \'e Typescript, proposto come alternativa a Dart, dal quale si differenzia per la sua sintassi molto simile a JavaScript.
Inoltre, per sfruttare a pieno le sue performance, non necessita di una particolare virtul machine poich\'e sfrutta quella integrata in tutti i browser per eseguire JavaScript.
Il nuovo framework Anglar2+, successore di AngularJS, scelse di appoggiarsi a Typescript per il vantaggi di avere performance indipendenti dal browser.


uno degli obiettivi dell'elaborato è stato quindi prevedere quale nuovo linguaggio potr\'a in futuro sostituire JavaScript, ma il fine principale era fornire un'analisi e un'indicazione delle strategie utilizzabili per lo sviluppo di un'applicazione web a pagina singola, affinchè qualunque utilizzatore possa avere una panoramica di questo mondo.