%%%%%%%%%%%%%%%%%%%%%%%%%%%%%%%%%%%%%%%%%12pt: grandezza carattere
%a4paper: formato a4
%openright: apre i capitoli a destra
%twoside: serve per fare un
%   documento fronteretro
%report: stile tesi (oppure book)
\documentclass[12pt,a4paper,openright,twoside]{report}
%
%%%%%%%%%%%%%%%%%%%%%%%%%%%%%%%%%%%%%%%%%libreria per scrivere in italiano
\usepackage[italian]{babel}
%
%%%%%%%%%%%%%%%%%%%%%%%%%%%%%%%%%%%%%%%%%libreria per accettare i caratteri
%   digitati da tastiera come è à
%   si può usare anche
%   \usepackage[T1]{fontenc}
%   però con questa libreria
%   il tempo di compilazione
%   aumenta
\usepackage[latin1]{inputenc}
%
%%%%%%%%%%%%%%%%%%%%%%%%%%%%%%%%%%%%%%%%%libreria per impostare il documento
\usepackage{fancyhdr}
%
%%%%%%%%%%%%%%%%%%%%%%%%%%%%%%%%%%%%%%%%%libreria per avere l'indentazione
%%%%%%%%%%%%%%%%%%%%%%%%%%%%%%%%%%%%%%%%%   all'inizio dei capitoli, ...
\usepackage{indentfirst}
%
%%%%%%%%%libreria per mostrare le etichette
%\usepackage{showkeys}
%
%%%%%%%%%%%%%%%%%%%%%%%%%%%%%%%%%%%%%%%%%libreria per inserire grafici
\usepackage{graphicx}
%
%%%%%%%%%%%%%%%%%%%%%%%%%%%%%%%%%%%%%%%%%libreria per utilizzare font
%   particolari ad esempio
%   \textsc{}
\usepackage{newlfont}
%
%%%%%%%%%%%%%%%%%%%%%%%%%%%%%%%%%%%%%%%%%librerie matematiche
\usepackage{amssymb}
\usepackage{amsmath}
\usepackage{latexsym}
\usepackage{amsthm}
%
\oddsidemargin=30pt \evensidemargin=20pt%impostano i margini
\hyphenation{sil-la-ba-zio-ne pa-ren-te-si}%serve per la sillabazione: tra parentesi 
%vanno inserite come nell'esempio le parole
%					   %che latex non riesce a tagliare nel modo giusto andando a capo.

%
%%%%%%%%%%%%%%%%%%%%%%%%%%%%%%%%%%%%%%%%%comandi per l'impostazione
%   della pagina, vedi il manuale
%   della libreria fancyhdr
%   per ulteriori delucidazioni
\pagestyle{fancy}\addtolength{\headwidth}{20pt}
\renewcommand{\chaptermark}[1]{\markboth{\thechapter.\ #1}{}}
\renewcommand{\sectionmark}[1]{\markright{\thesection \ #1}{}}
\rhead[\fancyplain{}{\bfseries\leftmark}]{\fancyplain{}{\bfseries\thepage}}
\cfoot{}
%%%%%%%%%%%%%%%%%%%%%%%%%%%%%%%%%%%%%%%%%
\linespread{1.3}                        %comando per impostare l'interlinea
%%%%%%%%%%%%%%%%%%%%%%%%%%%%%%%%%%%%%%%%%definisce nuovi comandi
%
\begin{document}
    \begin{titlepage}                       %crea un ambiente libero da vincoli
        %   di margini e grandezza caratteri:
        %   si pu\`o modificare quello che si
        %   vuole, tanto fuori da questo
        %   ambiente tutto viene ristabilito
        %
        \thispagestyle{empty}                   %elimina il numero della pagina
        \topmargin=6.5cm                        %imposta il margina superiore a 6.5cm
        \raggedleft                             %incolonna la scrittura a destra
        \large                                  %aumenta la grandezza del carattere
        %   a 14pt
        \em                                     %emfatizza (corsivo) il carattere
        Questa \`e la \textsc{Dedica}:\\
        ognuno pu\`o scrivere quello che vuole, \\
        anche nulla \ldots                      %\ldots lascia tre puntini
        \newpage                                %va in una pagina nuova
        %
        %%%%%%%%%%%%%%%%%%%%%%%%%%%%%%%%%%%%%%%%
        \clearpage{\pagestyle{empty}\cleardoublepage}
        %non numera l'ultima pagina sinistra
    \end{titlepage}
    \pagenumbering{roman}                   %serve per mettere i numeri romani
    \chapter*{Introduzione}                 %crea l'introduzione (un capitolo
    %   non numerato)
    %%%%%%%%%%%%%%%%%%%%%%%%%%%%%%%%%%%%%%%%%imposta l'intestazione di pagina
    \rhead[\fancyplain{}{\bfseries
    INTRODUZIONE}]{\fancyplain{}{\bfseries\thepage}}
    \lhead[\fancyplain{}{\bfseries\thepage}]{\fancyplain{}{\bfseries
    INTRODUZIONE}}
    %%%%%%%%%%%%%%%%%%%%%%%%%%%%%%%%%%%%%%%%%aggiunge la voce Introduzione
    %   nell'indice
    \addcontentsline{toc}{chapter}{Introduzione}
    Questa \`e l'introduzione.
    %%%%%%%%%%%%%%%%%%%%%%%%%%%%%%%%%%%%%%%%%non numera l'ultima pagina sinistra
    \clearpage{\pagestyle{empty}\cleardoublepage}
    \tableofcontents                        %crea l'indice
    %%%%%%%%%%%%%%%%%%%%%%%%%%%%%%%%%%%%%%%%%imposta l'intestazione di pagina
    \rhead[\fancyplain{}{\bfseries\leftmark}]{\fancyplain{}{\bfseries\thepage}}
    \lhead[\fancyplain{}{\bfseries\thepage}]{\fancyplain{}{\bfseries
    INDICE}}
    %%%%%%%%%%%%%%%%%%%%%%%%%%%%%%%%%%%%%%%%%non numera l'ultima pagina sinistra
    \clearpage{\pagestyle{empty}\cleardoublepage}
    \listoffigures                          %crea l'elenco delle figure
    %%%%%%%%%%%%%%%%%%%%%%%%%%%%%%%%%%%%%%%%%non numera l'ultima pagina sinistra
    \clearpage{\pagestyle{empty}\cleardoublepage}
    \listoftables                           %crea l'elenco delle tabelle
    %%%%%%%%%%%%%%%%%%%%%%%%%%%%%%%%%%%%%%%%%non numera l'ultima pagina sinistra
    \clearpage{\pagestyle{empty}\cleardoublepage}
    \chapter{Primo Capitolo}                %crea il capitolo
    %%%%%%%%%%%%%%%%%%%%%%%%%%%%%%%%%%%%%%%%%imposta l'intestazione di pagina
    \lhead[\fancyplain{}{\bfseries\thepage}]{\fancyplain{}{\bfseries\rightmark}}
    \pagenumbering{arabic}                  %mette i numeri arabi
    Questo \`e il primo capitolo.
    \section{Prima Sezione}                 %crea la sezione
    Questa \`e la prima sezione.

    Ora vediamo un elenco numerato:         %crea un elenco numerato
    \begin{enumerate}
        \item primo oggetto
        \item secondo oggetto
        \item terzo oggetto
        \item quarto oggetto
    \end{enumerate}

    \begin{figure}[h]                       %crea l'ambiente figura; [h] sta
        %   per here, cioè la figura va qui
        \begin{center}                          %centra nel mezzo della pagina
            %   la figura
            %\includegraphics[width=5cm]{figura.eps}%inserisce una figura larga 5cm
            %se si vuole usare va scommentata
            %
            %%%%%%%%%%%%%%%%%%%%%%%%%%%%%%%%%%%%%%%%%inserisce la legenda ed etichetta
            %   la figura con \label{fig:prima}
            \caption[legenda elenco figure]{legenda sotto la figura}\label{fig:prima}
        \end{center}
    \end{figure}

    \section{Seconda Sezione}
    Ora vediamo un elenco puntato:
    \begin{itemize}                         %crea un elenco puntato
        \item primo oggetto
        \item secondo oggetto
    \end{itemize}

    \section{Altra Sezione}
    Vediamo un elenco descrittivo:
    \begin{description}                     %crea un elenco descrittivo
        \item[OGGETTO1] prima descrizione;
        \item[OGGETTO2] seconda descrizione;
        \item[OGGETTO3] terza descrizione.
    \end{description}
    %%%%%%%%%%%%%%%%%%%%%%%%%%%%%%%%%%%%%%%%%crea una sottosezione
    \subsection{Altra SottoSezione}
    %%%%%%%%%%%%%%%%%%%%%%%%%%%%%%%%%%%%%%%%%crea una sottosottosezione
    \subsubsection{SottoSottoSezione}Questa sottosottosezione non viene
    numerata, ma \`e solo scritta in grassetto.
    \section{Altra Sezione}                 %crea una sottosezione
    Vediamo la creazione di una tabella; la tabella \ref{tab:uno}
    (richiamo il nome della tabella utilizzando la label che ho messo sotto):
    la facciamo di tre righe e tre colonne, la prima colonna
    ``incolonnata'' a destra (r) e le altre centrate (c):\\
    \begin{table}[h]                        %ambiente tabella
        %(serve per avere la legenda)
        \begin{center}                          %centra nella pagina la tabella
            \begin{tabular}{r|c|c}                  %tre colonne con righe verticali
                %   prodotte con |
                \hline \hline                           %inserisce due righe orizzontali
                $(1,1)$ & $(1,2)$ & $(1,3)$\\           %& separa le colonne e con
                \hline                                  %inserisce una riga orizzontale
                $(2,1)$ & $(2,2)$ & $(2,3)$\\           %  \\ va a capo
                \hline                                  %inserisce una riga orizzontale
                $(3,1)$ & $(3,2)$ & $(3,3)$\\
                \hline \hline
                %inserisce due righe orizzontali
            \end{tabular}
            \caption[legenda elenco tabelle]{legenda tabella}\label{tab:uno}
        \end{center}
    \end{table}
    \section{Altra Sezione}\label{sec:prova}%posso mettere le label anche
    %   alle section
    \subsection{Listati dei programmi}
    \subsubsection{Primo Listato}
    \begin{verbatim}
        In questo ambiente posso scrivere come voglio,
        lasciare gli spazi che voglio e non % commentare quando voglio
        e ci sarà scritto tutto.
        Quando lo uso è meglio che disattivi il Wrap del WinEdt
    \end{verbatim}
    %%%%%%%%%%%%%%%%%%%%%%%%%%%%%%%%%%%%%%%%%non numera l'ultima pagina sinistra
    \clearpage{\pagestyle{empty}\cleardoublepage}
    %%%%%%%%%%%%%%%%%%%%%%%%%%%%%%%%%%%%%%%%%per fare le conclusioni
    \chapter*{Conclusioni}
    %%%%%%%%%%%%%%%%%%%%%%%%%%%%%%%%%%%%%%%%%imposta l'intestazione di pagina
    \rhead[\fancyplain{}{\bfseries
    CONCLUSIONI}]{\fancyplain{}{\bfseries\thepage}}
    \lhead[\fancyplain{}{\bfseries\thepage}]{\fancyplain{}{\bfseries
    CONCLUSIONI}}
    %%%%%%%%%%%%%%%%%%%%%%%%%%%%%%%%%%%%%%%%%aggiunge la voce Conclusioni
    %   nell'indice
    \addcontentsline{toc}{chapter}{Conclusioni} Queste sono le
    conclusioni.\\
    In queste conclusioni voglio fare un riferimento alla
    bibliografia: questo \`e il mio riferimento \cite{K3,K4}.
    %%%%%%%%%%%%%%%%%%%%%%%%%%%%%%%%%%%%%%%%%imposta l'intestazione di pagina
    \renewcommand{\chaptermark}[1]{\markright{\thechapter \ #1}{}}
    \lhead[\fancyplain{}{\bfseries\thepage}]{\fancyplain{}{\bfseries\rightmark}}
    \appendix                               %imposta le appendici
    \chapter{Prima Appendice}               %crea l'appendice
    In questa Appendice non si \`e utilizzato il comando:\\
    %%%%%%%%%%%%%%%%%%%%%%%%%%%%%%%%%%%%%%%%%\verb"" è equivalente all'
    %   ambiente verbatim,
    %   ma si utilizza all'interno
    %   di un discorso.
    \verb"\clearpage{\pagestyle{empty}\cleardoublepage}", ed infatti
    l'ultima pagina 8 ha l'intestazione con il numero di pagina in
    alto.
    %%%%%%%%%%%%%%%%%%%%%%%%%%%%%%%%%%%%%%%%%imposta l'intestazione di pagina
    \rhead[\fancyplain{}{\bfseries \thechapter \:Prima Appendice}]
    {\fancyplain{}{\bfseries\thepage}}
    \chapter{Seconda Appendice}             %crea l'appendice
    %%%%%%%%%%%%%%%%%%%%%%%%%%%%%%%%%%%%%%%%%imposta l'intestazione di pagina
    \rhead[\fancyplain{}{\bfseries \thechapter \:Seconda Appendice}]
    {\fancyplain{}{\bfseries\thepage}}
    \begin{thebibliography}{90}             %crea l'ambiente bibliografia
        \rhead[\fancyplain{}{\bfseries \leftmark}]{\fancyplain{}{\bfseries
        \thepage}}
        %%%%%%%%%%%%%%%%%%%%%%%%%%%%%%%%%%%%%%%%%aggiunge la voce Bibliografia
        %   nell'indice
        \addcontentsline{toc}{chapter}{Bibliografia}
        %%%%%%%%%%%%%%%%%%%%%%%%%%%%%%%%%%%%%%%%%provare anche questo comando:
        %%%%%%%%%%%\addcontentsline{toc}{chapter}{\numberline{}{Bibliografia}}
        \bibitem{K1} Primo oggetto bibliografia.
        \bibitem{K2} Secondo oggetto bibliografia.
        \bibitem{K3} Terzo oggetto bibliografia.
        \bibitem{K4} Quarto oggetto bibliografia.
    \end{thebibliography}
    %%%%%%%%%%%%%%%%%%%%%%%%%%%%%%%%%%%%%%%%%non numera l'ultima pagina sinistra
    \clearpage{\pagestyle{empty}\cleardoublepage}
    \chapter*{Ringraziamenti}
    \thispagestyle{empty}
    Qui possiamo ringraziare il mondo intero!!!!!!!!!!\\
    Ovviamente solo se uno vuole, non \`e obbligatorio.
\end{document}
